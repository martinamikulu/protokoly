\documentclass[12pt,a4paper]{article}
\usepackage[utf8]{inputenc}
\usepackage{amsmath}
\usepackage{amsfonts}
\usepackage{amssymb}
\usepackage{graphicx}
\usepackage[left=2.00cm, right=2.00cm, top=2.00cm, bottom=2.00cm]{geometry}
\usepackage[czech]{babel}
\usepackage{upgreek}

\def\ri#1{\mathrm{#1}}
\def\jd#1{\ifmmode\,\mathrm{#1}\else\,\textrm{#1}\fi}

\begin{document}
\begin{center}
\includegraphics*[width=\linewidth]{rozdel}
\end{center}
\section*{Princip úlohy}
Rozdělovací poměr, $P$, ve dvou nemísitelných rozpouštědlech je definován jako poměr analytických koncentrací rozpuštěné látky v jednotlivých rozpouštědlech. Pro systém toluen--voda--kyselina benzoová je rozdělovací poměr, $P$, definován jako
\begin{equation}
	P = \dfrac{c_\ri{t}}{c_\ri{v}},
\end{equation}
kde $c_\ri{t}$ je koncentrace kyseliny benzoové v toluenu a $c_\ri{v}$ je koncentrace kyseliny benzoové ve vodě. Koncentraci kyseliny benzoové ve vodě lze stanovit titrací kyseliny benzoové hydroxidem sodným. Hydroxid sodný reaguje s kyselinou benzoovou ve stechiometrickém poměru 1:1 podle rovnice
\begin{equation}
	\ri{C_6H_5COOH + NaOH}\rightarrow \ri{C_6H_5COONa + H_2O}.
\end{equation}
Ze zjištěné koncentrace kyseliny benzoové ve vodě můžeme přes látková množství spočítat koncentraci kyseliny benzoové v toluenu. Rozdělovací poměr, $P$, závisí na celkovém látkovém množství kyseliny benzoové v systému a není tak konstantní.\\
Rozdělovací rovnováha je charakterizovaná Nernstovým rozdělovacím koeficientem $k_\ri{r}$ jako poměr aktivit rozpuštěné látky v obou rozpouštědlech. V případě velmi zředěných roztoků lze tyto aktivity ztotožnit s molárními koncentracemi rozpuštěné látky v obou rozpouštědlech jako 
\begin{equation}
	k_\ri{r} = \dfrac{a_\ri{A, 1}}{a_\ri{A, 2}} = \approx \dfrac{c_\ri{A, 1}}{c_\ri{A, 2}}, 
\end{equation}
kde $a_\ri{A, 1}$ a $a_\ri{A, 2}$ jsou aktivity rozpuštěné látky A v rozpouštědlech 1 a 2 a $c_\ri{A, 1}$ a $c_\ri{A, 2}$ jsou koncentrace rozpuštěné látky A v rozpouštědlech 1 a 2. Tato rovnováha však platí pouze v případě, že se rozpuštěná látka nachází v systému pouze v jedné formě. Protože kyselina benzoová v toluenové fázi dimeruje, je rozdělovací koeficient definován jako poměr koncentrací monomeru v toluenové a vodné fázi. Disociaci kyseliny benzoové ve vodné fázi můžeme zanedbat a pro Nernstův rozdělovací koeficient dostáváme vztah
\begin{equation}
	k_\ri{r} = \dfrac{c_\ri{m, t}}{c_\ri{v}},
\end{equation}
kde $c_\ri{m, t}$ je koncentrace kyseliny benzoové v toluenu, kterou vyjádříme pomocí konstanty dimerizace jako 
\begin{equation}
	c_\ri{m, t} = \sqrt{\dfrac{c_\ri{d, t}}{K_\ri{D}}},
\end{equation}
kde $c_\ri{d, t}$ je koncentrace dimeru v toluenu a $K_\ri{D}$ je konstanta dimerizace kyseliny benzoové v toluenu. Analytická koncentrace kyseliny benzoové v toluenu je daná jako součet koncentrace monomeru a dvojnásobku koncentrace dimeru, protože jeden dimer je tvořen dvěma molekulami kyseliny benzoové. Pro analytickou koncentraci kyseliny benzoové v toluenu, $c_\ri{t}$, tedy platí
\begin{equation}
	c_\ri{t} = c_\ri{m, t} + 2\,c_\ri{d, t}.
\end{equation}
Vyjádřením $c_\ri{d, t}$ z rovnice (6) a jeho dosazením do rovnice (5) získáme vztah
\begin{equation}
	c_\ri{m, t} = \sqrt{\dfrac{c_\ri{t}-c_\ri{m, t}}{2\,K_\ri{D}}}.
\end{equation}
Protože v toluenu téměř všechna kyselina benzoová dimeruje můžeme v rovnici (7) $c_\ri{m, t}$ oproti $c_\ri{t}$ zanedbat a dosazením do rovnice (4) získáme vztah
\begin{equation}
	k_\ri{r} = \dfrac{\sqrt{\dfrac{c_\ri{t}}{2\,K_\ri{D}}}}{c_\ri{v}},
\end{equation}
ze kterého vyjádříme vztah
\begin{equation}
	k_\ri{r}\sqrt{2\,K_\ri{D}} =  \dfrac{\sqrt{c_\ri{t}}}{c_\ri{v}}.
\end{equation}
Protože $k_\ri{r}$ a $K_\ri{D}$ jsou konstanty, je i poměr $\dfrac{\sqrt{c_\ri{t}}}{c_\ri{v}}$ konstantní pro všechny měřené vzorky.\\
Stupeň disociace $\alpha$ pro kyselinu benzoovou ve vodě spočítáme podle vztahu
\begin{equation}
	\alpha = \sqrt{\dfrac{10^{-\ri{p}K_\ri{A}}}{c_\ri{v, rel}}},
\end{equation}
kde p$K_\ri{A}$ je logaritmický tvar disociační konstanty kyseliny benzoové, podle návodu k úloze roven 4,2 a $c_\ri{v, rel}$ je relativní koncentrace kyseliny benzoové ve vodě. Tuto relativní koncentraci získáme podělením koncentrace kyseliny benzoové ve vodě, $c_\ri{v}$, standardní molární koncentrací $c^\ominus = 1\jd{mol\cdot dm^{-3}}.$
\section*{Postup}
Nejprve byla do 4 Erlenmayerových baněk odvážena kyselina benzoová. Přesné navážky jsou uvedeny v tabulce č. 1. Do každé baňky bylo napipetováno 25,0\jd{ml} vody a 10,0\jd{ml} toluenu. Baňky byly uzavřeny a upevněny do automatické třepačky, na které byly třepány při frekvenci $500\jd{min^{-1}}$ po dobu 30\jd{min}. Poté byl obsah každé baňky přelit do dělicí nálevky a bylo počkáno cca 40\jd{min}, než se hladina mezi fázemi ustálila. Poté byla z každého vzorku do uzavíratelné lahvičky odpuštěna vodná fáze. Poté bylo vždy 5,00\jd{ml} vodné fáze každého vzorku titrováno 0,0200M roztokem hydroxidu sodného v přítomnosti fenolftaleinu. Pro každý vzorek byla titrace provedena dvakrát. Získané spotřeby hydroxidu sodného $V_\ri{NaOH, 1}$ a $V_\ri{NaOH, 2}$ jsou uvedeny v tabulce č. 1. Pro každý vzorek pak byla spočítána průměrná spotřeba hydroxidu sodného $V_\ri{NaOH}$, která je taktéž uvedena v tabulce č. 1.
\begin{center}
	\noindent Tab. č. 1: Navážky kyseliny benzoové v jednotlivých vzorcích a spotřeby roztoku NaOH pro tyto vzorky\\
	\begin{tabular}{c|c|c|c|c}
		Vzorek č. & $m_\ri{b}$/g & $V_\ri{NaOH, 1}$/ml & $V_\ri{NaOH, 2}$/ml & $V_\ri{NaOH}$/ml\\
		\hline
		1 & 0,1008 & 2,08 & 2,08 & 2,08\\
		2 & 0,1499 & 2,64 & 2,72 & 2,68\\
		3 & 0,2002 & 3,20 & 3,14 & 3,17\\
		4 & 0,2501 & 3,58 & 3,56 & 3,57\\
	\end{tabular}
\end{center}
\section*{Vyhodnocení výsledků}	
Látkové množství kyseliny benzoové v systému, $n_\ri{benz}$, spočítáme podle vzorce
\begin{equation}
	n_\ri{benz} = \dfrac{m}{M},
\end{equation}
kde $m$ je navážka kyseliny benzoové a $M$ je molární hmotnost kyseliny benzoové, která je podle [1] rovna $122,124\jd{g\cdot mol^{-1}}$. Pro vzorek č. 1 získáme po dosazení výpočet
$$n=\dfrac{0,1008}{122,124} = 0,0008254\jd{mol}.$$
Hodnoty $n$ pro všechny vzorky jsou uvedeny v tabulce č. 2.\\
Koncentraci kyseliny benzoové ve vodné fázi spočítáme jako 
\begin{equation}
	c_\ri{v}\cdot V_\ri{v} = c_\ri{NaOH}\cdot V_\ri{NaOH} \Rightarrow c_\ri{v} = \dfrac{c_\ri{NaOH}\cdot V_\ri{NaOH}}{V_\ri{v}},
\end{equation}
kde $c_\ri{v}$ je koncentrace kyseliny benzoové ve vodné fázi, $V_\ri{v}$ je titrovaný objem vodné fáze, tedy 5,00\jd{ml} a $c_\ri{NaOH}$ je molární koncentrace odměrného roztoku hydroxidu sodného, tedy $0,0200\jd{mol\cdot dm^{-3}}$. $V_\ri{NaOH}$ je průměrná spotřeba hydroxidu sodného, který pro vzorek č. 1 činí 2,08\jd{ml}. Dosazením těchto hodnot získáme výpočet koncentrace kyseliny benzoové ve vodné fázi vzorku č. 1, tedy
$$c_\ri{v} = \dfrac{0,0200\cdot 2,08}{5,00} = 0,00832\jd{mol\cdot dm^{-3}}$$
Hodnoty $c_\ri{v}$ pro všechny vzorky jsou uvedeny v tabulce č. 2.\\
Z hodnot koncentrace kyseliny benzoové ve vodné fázi spočítáme látkové množství kyseliny benzoové ve vodné fázi, $n_\ri{v}$ podle vzorce 
\begin{equation}
	n_\ri{v} = c_\ri{v}\cdot V_\ri{vf},
\end{equation}
kde $V_\ri{vf}$ je objem vodné fáze, tedy $0,0250\jd{dm^3}$. Koncentrace kyseliny benzoové v toluenu, $c_\ri{t}$ je daná jako podíl látkového množství kyseliny benzoové v toluenu, $n_\ri{t}$, a objemu toluenové fáze, $V_\ri{tf}$.
\begin{equation}
	c_\ri{t} = \dfrac{n_\ri{t}}{V_\ri{tf}}
\end{equation}
Látkové množství kyseliny benzoové v toluenu získáme jako rozdíl celkového látkového množství kyseliny benzoové v systému, $n_\ri{benz}$, a látkového množství kyseliny benzoové ve vodě, $n_\ri{v}$.
\begin{equation}
	n_\ri{t} = n_\ri{benz} - n_\ri{v}
\end{equation}
Dosazením rovnice (13) do rovnice (15) a následně dosazením takto upravené rovnice (15) do rovnice (14) získáme pro výpočet koncentrace kyseliny benzoové v toluenové fázi rovnici
\begin{equation}
	c_\ri{t} = \dfrac{n_\ri{benz} - c_\ri{v}\cdot V_\ri{f}}{V_\ri{tf}}.
\end{equation}
Po dosazení získáme pro vzorek č. 1 výpočet
$$c_\ri{t} = \dfrac{0,0008254 - 0,00832\cdot 0,0250}{0,0100} = 0,0617\jd{mol\cdot dm^{-3}}$$
Hodnoty $c_\ri{t}$ pro jednotlivé vzorky jsou v tabulce č. 2. Podle rovnic (1) a (10) byl pro každý vzorek spočítán rozdělovací poměr, $P$, a stupeň disociace kyseliny benzoové ve vodě, $\alpha$. Taktéž byl pro každý vzorek spočítán poměr $\dfrac{\sqrt{c_\ri{t}}}{c_\ri{v}}$. Pro vzorek č. 1 získáváme po dosazení výpočty
$$P=\dfrac{0,0617}{0,00832} = 7,42,$$
$$\dfrac{\sqrt{c_\ri{t}}}{c_\ri{v}} = \dfrac{\sqrt{0,0617}}{0,00832} = 29,9\jd{mol^{-\frac{1}{2}}\cdot dm^{\frac{3}{2}}}$$
a
$$\alpha = \sqrt{\dfrac{10^{-4,2}}{0,00832}} = 0,087$$
\begin{center}
	\noindent Tab. č. 2: Vypočítané hodnoty pro jednotlivé vzorky\\
	\begin{tabular}{c|c|c|c|c|c|c}
		Vzorek č. & $n_\ri{benz}$/mol & $c_\ri{v}/\jd{mol\cdot dm^{-3}}$ & $c_\ri{t}/\jd{mol\cdot dm^{-3}}$ & $P$ & $\dfrac{\sqrt{c_\ri{t}}}{c_\ri{v}}/\jd{mol^{-\frac{1}{2}}\cdot dm^{\frac{3}{2}}}$ & $\alpha$/\%\\
		\hline
		1 & 0,0008254 & 0,00832 & 0,0617 & 7,42 & 29,9 & 8,7\\
		2 & 0,001227 & 0,0107 & 0,0959 & 8,95 & 28,9 & 7,7\\
		3 & 0,001639 & 0,0127 & 0,132 & 10,4 & 28,7 & 7,1\\
		4 & 0,002048 & 0,0143 & 0,169 & 11,8 & 28,8 & 6,6\\
	\end{tabular}
\end{center}
\section*{Diskuse}
Přesnost zjištěných rozdělovacích poměrů a poměrů $\dfrac{\sqrt{c_\ri{t}}}{c_\ri{v}}$ může být značně ovlivněna disociací kyseliny benzoové ve vodě, jelikož její stupeň disociace se pohybuje mezi 6-9\jd{\%}. Pro přesnější definici Nernstova rozdělovacího koeficientu by tak měla být odečtena koncentrace disociované formy kyseliny benzoové od celkové koncentrace kyseliny benzoové ve vodné fázi.\\
Zjištěné hodnoty poměru $\dfrac{\sqrt{c_\ri{t}}}{c_\ri{v}}$ jsou v rámci chyby měření stejné, liší se pouze hodnota poměru u prvního vzorku. Tato odlišnost mohla být způsobena nepřesným pipetováním nebo titrováním vzorku. Hodnota rozdělovacího poměru, $P$, podle očekávání roste s rostoucí navážkou.
\newpage
\section*{Závěr}
Pro systém toluen--voda--kyselina benzoová byly pro známé hmotnosti kyseliny benzoové stanoveny hodnoty rozdělovacího koeficientu, $P$, poměru $\dfrac{\sqrt{c_\ri{t}}}{c_\ri{v}}$ a stupně disociace $\alpha$. Tyto hodnoty jsou uvedeny v tabulce č. 3.
\begin{center}
	\noindent Tab. č. 3: Výsledné hodnoty $P$, $\dfrac{\sqrt{c_\ri{t}}}{c_\ri{v}}$ a $\alpha$ pro jednotlivé navážky\\
	\begin{tabular}{c|c|c|c|c}
		Vzorek č. & $m_\ri{benz}$/g & $P$ & $\dfrac{\sqrt{c_\ri{t}}}{c_\ri{v}}/\jd{mol^{-\frac{1}{2}}\cdot dm^{\frac{3}{2}}}$ & $\alpha$/\%\\
		\hline
		1 & 0,1008 & 7,42 & 29,9 & 8,7\\
		2 & 0,1499 & 8,95 & 28,9 & 7,7\\
		3 & 0,2002 & 10,4 & 28,7 & 7,1\\
		4 & 0,2501 & 11,8 & 28,8 & 6,6\\
	\end{tabular}
\end{center}
\section*{Zdroje}
[1] J. Vohlídal, A. Julák, K. Štulík: Chemické a analytické tabulky, Praha 1999
\end{document}