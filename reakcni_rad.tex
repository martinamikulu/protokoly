\documentclass[12pt,a4paper]{article}
\usepackage[utf8]{inputenc}
\usepackage{amsmath}
\usepackage{amsfonts}
\usepackage{amssymb}
\usepackage{graphicx}
\usepackage[left=2.00cm, right=2.00cm, top=2.00cm, bottom=2.00cm]{geometry}
\usepackage[czech]{babel}
\usepackage{upgreek}

\def\ri#1{\mathrm{#1}}
\def\jd#1{\ifmmode\,\mathrm{#1}\else\,\textrm{#1}\fi}

\begin{document}
\begin{center}
\includegraphics*[width=\linewidth]{RR_zahlavi}
\end{center}
\section*{Princip úlohy}
\noindent Dílčí reakční řády byly zjišťovány pro reaktanty reakce popsané následující rovnicí
\begin{equation}
	\ri{IO_3^-\,+\,5\,I^-\,+\,6\,H_3O^+} \rightarrow \ri{3\,I_2\,+\,9\,H_2O}.
\end{equation}
Vznikající jód je redukován podle rovnice
\begin{equation}
	\ri{I_2\,+\,SO_3^-\,+\,3\,H_2O} \rightarrow \ri{2\,I^-\,+\,2\,H_3O^+\,+\,SO_4^\ri{2-}}
\end{equation}
do úplného spotřebování siřičitanu sodného.
Rychlost reakce je definovaná jako 
\begin{equation}
	v = \dfrac{1}{\nu_\ri{i}}\dfrac{\ri{d}c_\ri{i}}{\ri{d}t},
\end{equation}
kde $c_\ri{i}$ je okamžitá koncentrace i--té složky reakce v čase reakce t a $\nu_\ri{i}$ je stechiometrický koeficient této i--té složky. Rychlost reakce můžeme také popsat pomocí okamžitých koncentrací všech reaktantů $c_\ri{A}$, $c_\ri{B}$ a $c_\ri{C}$ jako 
\begin{equation}
	v = kc_\ri{A}^\alpha c_\ri{B}^\beta c_\ri{C}^\gamma,
\end{equation}
kde \textit{k} je rychlostní konstanta reakce a $\alpha$, $\beta$ a $\gamma$ jsou dílčí reakční řády jednotlivých reaktantů. Počáteční rychlost reakce $v_0$ lze tedy vyjádřit jako  
\begin{equation}
v_0 = kc_\ri{A, 0}^\alpha c_\ri{B, 0}^\beta c_\ri{C, 0}^\gamma, 
\end{equation}
kde $c_\ri{A, 0}$, $c_\ri{B, 0}$ a $c_\ri{C, 0}$ jsou počáteční koncentrace jednotlivých reaktantů.\\
Metoda počátečních rychlostí je použitelná pro stupeň konverze $\alpha<0,05$. Počáteční rychlost reakce pak pro tento stupeň konverze můžeme aproximovat za průměrnou rychlost jako
\begin{equation}
	v_0 \cong \dfrac{1}{\nu_\ri{i}}\dfrac{\Delta c_\ri{i}}{t},
\end{equation}
kde $\Delta c_\ri{i}$ je předem definovaná koncentrace produktu, který při reakci vznikne za čas t. Spojením rovnic (5) a (6) získáme rovnici
\begin{equation}
	\dfrac{1}{\nu_\ri{i}}\dfrac{\Delta c_\ri{i}}{t} = kc_\ri{A, 0}^\alpha c_\ri{B, 0}^\beta c_\ri{C, 0}^\gamma.
\end{equation}
Metoda počátečních rychlostí spočívá v~tom, že postupně měníme koncentraci jednoho reaktantu při konstantních koncentracích všech ostatních reaktantů, a sledujeme čas \textit{t}, za který v~reakční směsi vznikne předem definovaná koncentrace produktu $\Delta c_\ri{i}$. V případě tohoto konkrétního stanovení je předem definovaná koncentrace produktu taková koncentrace jodu, která vytvoří s~přítomným škrobem modré zbarvení v přítomnosti vždy stejného množství siřičitanu sodného jako redukčního činidla. Protože $\Delta c_\ri{i}$ je konstantní pro všechny měření, můžeme ji zahrnout do konstanty k rychlostní konstantě \textit{k}. Stejně tak můžeme do této konstanty zahrnout i neměnné koncentrace neměřených reaktantů a stechiometrický koeficient produktu. Tím dostaneme rovnici
\begin{equation}
	\dfrac{1}{t} = \ri{konst.}\cdot c_\ri{A, 0}^\alpha,
\end{equation}
jejímž zlogaritmováním získáme
\begin{equation}
	\log \dfrac{1}{t} = konst. + \alpha \cdot \log c_\ri{A, 0}.
\end{equation}
Vynesením závislosti $\log \dfrac{1}{t}$ na $\log c_\ri{A, 0}$ a proložením této závislosti přímkou získáme rovnici, jejíž koeficient lineárního členu je dílčí reakční řád sledovaného reaktantu. \\
Stupeň konverze $\alpha$ je definován jako
\begin{equation}
	\alpha = \dfrac{n_\ri{i, t}}{n_\ri{i, 0}}, 
\end{equation}
kde $n_\ri{i, 0}$ je počáteční látkové množství reaktantu i a $n_\ri{i, t}$ je zreagované látkové množství reaktantu i v čase t. V případě této reakce ze stechiometrie platí 
\begin{equation}
	n_\ri{I^-, t} = \dfrac{5}{3}\cdot c_\ri{SO_3^\ri{2-}}\cdot V_\ri{SO_3^\ri{2-}},
\end{equation}
kde $c_\ri{SO_3^\ri{2-}}$ je koncentrace zásobního roztoku siřičitanu sodného a  $V_\ri{SO_3^\ri{2-}}$ je objem zásobního roztoku siřičitanu sodného.
\section*{Postup}
\noindent Nejprve byl připraven zásobní roztok siřičitanu sodného. Podle zadání měl být připraven roztok o koncentraci $5,00\cdot 10^{-3}\jd{mol\cdot dm{-3}}$, protože byl však pro výpočet navážky použit omylem molární hmotnost heptahydrátu siřičitanu sodného, přestože roztok byl připravován z bezvodého siřičitanu sodného, bylo nutné koncentraci přepočítat. Návažka heptahydrátu byla spočítána následovně:
\begin{equation}
	m_\ri{Na_2SO_3.7H_2O} = c\cdot V\cdot M_\ri{Na_2SO_3.7H_2O}
\end{equation}
$$m_\ri{Na_2SO_3.7H_2O} = 5,00\cdot 10^{-3}\cdot 0,500\cdot 252,16 = 0,603\jd{g}$$
Na analytických vahách bylo naváženo 0,6304\jd{g} pevného bezvodého siřičitanu sodného o $M=126,04\jd{g\cdot mol^\ri{-1}}$, jehož rozpuštěním v 500\jd{ml} vody byl připraven roztok o koncentraci $1,00\cdot 10^\ri{-2}\jd{mol\cdot dm^\ri{-3}}$. Koncentrace zásobního roztoku byla spočítána podle následujících výpočtů:
\begin{equation}
	c = \dfrac{n}{V}
\end{equation}
\begin{equation}
	n = \dfrac{m}{M}
\end{equation}
\begin{equation}
	c = \dfrac{m}{V\cdot M}
\end{equation} 
$$c = 0,0100\jd{mol\cdot dm^\ri{-3}}$$
Následně byly připraveny pufry z 500mM zásobních roztoků octanu sodného a kyseliny octové podle tabulky č. 1 a bylo změřeno jejich pH na zkalibrovaném pH metru. Pro kalibraci byly použity pufry o pH = 7,00 a 4,01.
\begin{center}
	\noindent Tab. č. 1: Složení pufrů\\
	\begin{tabular}{c|c|c|c}
		Označení pufru & V ($\ri{CH_3COONa}$)/ml & V ($\ri{CH_3COOH}$)/ml & pH\\
		\hline
		1 & 25,0 & 25,0 & 4,63\\
		2 & 20,0 & 40,0 & 4,33\\
		4 & 10,0 & 40,0 & 4,02\\
		8 & 6,0 & 48,0 & 3,74\\
	\end{tabular}
\end{center} 
Následně byly připraveny roztoky I a I(1) až I(8) obsahující jodičnan draselný a roztoky II(10) až II(20) obsahující jodid draselný podle tabulek č. 2 a 3 a doplněny vodou na objem 50\jd{ml}.
\begin{center}
	\noindent Tab. č. 2: Složení roztoků obsahujících jodičnan draselný\\
	\begin{tabular}{c|c|c|c|c}
		Označení roztoku & V ($25\ri{mM} \ri{KIO_3}$)/ml & V ($\ri{CH_3COONa}$)/ml & V ($\ri{CH_3COOH}$)/ml & 40,0\jd{ml} pufru č.\\
		\hline
		I & 5,0 & 10,0 & 20,0 & -\\
		I(1) & 5,0 & - & - & 1\\
		I(2) & 5,0 & - & - & 2\\
		I(4) & 5,0 & - & - & 4\\
		I(8) & 5,0 & - & - & 8\\
	\end{tabular}
\end{center} 
\begin{center}
	\noindent Tab. č. 3: Složení roztoků obsahujích jodid draselný\\
	\begin{tabular}{c|c|c|c}
		Označení roztoku & V ($250\ri{mM} \ri{KI}$)/ml & V ($\ri{Na_2SO_3}$)/ml & V (roztok škrobu)/ml\\
		\hline
		II(10) & 10,0 & 15,0 & 4,0\\
		II(12) & 12,0 & 15,0 & 4,0\\
		II(14) & 14,0 & 15,0 & 4,0\\
		II(16) & 16,0 & 15,0 & 4,0\\
		II(18) & 18,0 & 15,0 & 4,0\\
		II(20) & 20,0 & 15,0 & 4,0\\
	\end{tabular}
\end{center}
Po připravení roztoků byly smíchány postupně vždy jednotlivé roztoky II(10) až II(20) s vždy s roztokem I tak, že roztok I byl přelit do kádinky s magnetickým míchadlem, tato kádinka byla postavena na magnetickou míchačku, za stálého míchání byl přidán konkrétní roztok II a ve stejném okamžiku byly zapnuty stopky. Poté byl měřen čas uplynulý od smíchání roztoků po zmodrání reakční směsi, během kterého byla směs stále míchána na magnetické míchačce. Stejným způsobem pak byl měřen čas pro reakce roztoků I(1) až I(8) vždy s roztokem II(10).
\section*{Vyhodnocení výsledků}	
\subsection*{Stanovení dílčího reakčního řádu a výpočet stupně konverze jodidových \hbox{aniontů}}
Koncentrace jodidových aniontů $c_\ri{I^-}$ v reakční směsi byla spočítána podle rovnice
\begin{equation}
	c_\ri{I^-} = \dfrac{c_\ri{KI, zas}\cdot V_\ri{KI, zas}}{V_\ri{smes}},
\end{equation}
kde $c_\ri{KI, zas}$ je koncentrace zásobního roztoku KI, $V_\ri{KI, zas}$ je pipetovaný objem zásobního roztoku KI a $V_\ri{smes}$ je objem reakční směsi vzniklé slitím dvou roztoků o objemech 50\jd{ml}, tedy 100\jd{ml}. Koncentrace jodidových aniontů v reakční směsi při stanovení roztoku II(10) je tedy spočítána jako
$$c_\ri{I^-} = \dfrac{0,2500\jd{mol\cdot dm^\ri{-3}}\cdot 10,0\jd{ml}}{100,0\jd{ml}} \Rightarrow c_\ri{I^-} = 0,0250\jd{mol\cdot dm^\ri{-3}}$$
Stupeň konverze jodidů $\alpha_\ri{I^-}$ se na základě rovnic (10) a (11) spočítá jako
\begin{equation}
	\alpha_\ri{I^-} = \dfrac{5\cdot c_\ri{SO_3^\ri{2-}}\cdot V_\ri{SO_3^\ri{2-}}}{3\cdot c_\ri{KI, zas}\cdot V_\ri{KI, zas}},
\end{equation}
pro výchozí objem KI 10\jd{ml} je stupeň konverze 
$$\alpha_\ri{I^-} = \dfrac{5\cdot 0,0100\jd{mol\cdot dm^{-3}} \cdot 15,0\jd{ml}}{3\cdot 0,2500\jd{mol\cdot dm^{-3}} \cdot 10,0\jd{ml}} \Rightarrow \alpha_\ri{I^-} = 0,100$$
\begin{center}
	\noindent Tab. č. 4: Naměřené časy při různých koncentracích jodidu draselného
	\begin{tabular}{c|c|c|c}
		Označení roztoku & $c_\ri{I^-}/\ri{mol\cdot dm^\ri{-3}}$ & $\alpha_\ri{I^-}$ & \textit{t}/s\\
		\hline
		II(10) & 0,0250 & 0,100 & 217,15\\
		II(12) & 0,0300 & 0,0833 & 187,04\\
		II(14) & 0,0350 & 0,0714 & 146,11\\
		II(16) & 0,0400 & 0,0625 & 120,25\\
		II(18) & 0,0450 & 0,0556 & 98,88\\
		II(20) & 0,0500 & 0,0500 & 74,38\\
	\end{tabular}
\end{center}
Následně byla v programu Origin do grafu č.1 vynesena závislost $\log \dfrac{1}{t}$ na $\log c_\ri{I^-}$. Touto závislostí byla proložena přímka, jejíž směrnice, v grafu označená jako Slope, odpovídá dílčímu reakčnímu řádu jodidů
\begin{equation}
	\alpha_\ri{I^-} = 1,53 \pm 0,13.
\end{equation}
\subsection*{Stanovení dílčího reakčního řádu oxoniových kationtů}
\begin{center}
	\noindent Tab. č. 5: Naměřené časy při různých pH pufru v roztoku I\\
	\begin{tabular}{c|c|c}
		Označení roztoku & pH & t/s\\
		\hline
		I(1) & 4,63 & 514,52\\
		I(2) & 4,33 & 211,33\\
		I(4) & 4,02 & 58,94\\
		I(8) & 3,74 & 15,29\\
	\end{tabular}
\end{center}
Definice pH je
\begin{equation}
\ri{pH} = -\log a_\ri{H_3O^+},
\end{equation}
kde $a_\ri{H_3O^+}$ je aktivita oxoniových kationtů definována jako součin koncentrace $c_\ri{H_3O^+}$ a aktivitního koeficientu $\gamma_\ri{H_3O^+}$
\begin{equation}
 a_\ri{H_3O^+}=\gamma_\ri{H_3O^+} \cdot c_\ri{H_3O^+}.
\end{equation} 
Při stanovení zanedbáme vliv iontové síly roztoku a předpokládáme, že aktivitní koeficient oxoniových kationtů $\gamma_\ri{H_3O^+}=1$ a tedy můžeme logaritmus koncentrace stanovovaných oxoniových kationtů vyjádřit jako
\begin{equation}
	\log c_\ri{H_3O^+} = -\ri{pH}.
\end{equation}
Na základě rovnice (20) byla v programu origin vynesena závislost $\log \dfrac{1}{t}$ na $-$pH. Touto závislostí byla proložena přímka, jejíž směrnice, v grafu označená jako Slope, odpovídá dílčímu reakčnímu řádu oxoniových kationtů
\begin{equation}
\alpha_\ri{H_3O^+} = 1,72 \pm 0,13.
\end{equation}
\section*{Diskuse}
Záporné dekadické logaritmy koncentrací byly do grafu omylem vyneseny pro poloviční objem, jelikož jsme si neuvědomili, že při slití obou roztoků se jejich objem sečte a tím se změní i koncentrace iontů v reakční směsi. Tento omyl se ale ve výsledku neprojevil, protože při uvažování správného objemu by byly hodnoty koncentrace iontů v reakční směsi poloviční a hodnoty na ose nezávislé proměnné by tak byly posunuty směrem k zápornějším hodnotám. Směrnice přímky ale zůstala nezměněná. Správně vypočítané hodnoty koncentrací jsou uvedeny v tabulce č. 4.\\
Zjištěné reakční řády mohou být zatíženy nepřesností při manuálním měření času nebo při pipetování při přípravě roztoků. Nejvýraznější chybu však přináší zjištění, že podmínka pro použití metody počátečních rychlostí nevyhovuje vypočítaným stupňům konverze jodidů, které téměř všechny přesahují 5\jd{\%}. Tyto vysoké stupně konverze jsou způsobeny chybnou přípravou zásobního roztoku siřičitanu sodného, jehož koncentraci jsme omylem připravili dvojnásobnou než měla být, protože jsme pro výpočet navážky použili chybně místo molární hmotnosti bezvodého siřičitanu sodného molární hmotnost heptahydrátu siřičitanu sodného. Jednotlivé reakce proto běžely příliš dlouho na to, aby byla metoda počátečních rychlostí použitelná.
\section*{Závěr}
Dílčí reakční řád jodidových aniontů je 1,53$\pm$0,13. Dílčí reakční řád oxoniových kationtů je 1,72$\pm$0,13.
\end{document}