\documentclass[12pt,a4paper]{article}
\usepackage[utf8]{inputenc}
\usepackage{amsmath}
\usepackage{amsfonts}
\usepackage{amssymb}
\usepackage{graphicx}
\usepackage[left=2.00cm, right=2.00cm, top=2.00cm, bottom=2.00cm]{geometry}
\usepackage[czech]{babel}
\usepackage{upgreek}

\def\ri#1{\mathrm{#1}}
\def\jd#1{\ifmmode\,\mathrm{#1}\else\,\textrm{#1}\fi}

\begin{document}
\begin{center}
\includegraphics*[width=\linewidth]{vodivost}
\end{center}
\section*{Princip úlohy}
Schopnost roztoku vést elektrický proud je způsobena přítomností nabitých částic. Protože slabé elektrolyty v roztoku disociují pouze částečně, je jejich vodivost závislá na stupni disociace, $\alpha$ podle vztahu
\begin{equation}
	\alpha = \dfrac{\Lambda}{\Lambda_\infty},
\end{equation}
kde $\Lambda$ je molární vodivost měřeného roztoku a $\Lambda_\infty$ je limitní molární vodivost roztoku, tedy vodivost nekonečně zředěného roztoku. Tato limitní molární vodivost se spočítá jako součet limitních molárních vodivostí iontů v roztoku, tedy v případě roztoku kyseliny propionové jako
\begin{equation}
	\Lambda_\infty = \Lambda_\ri{\infty, H^+} + \Lambda_\ri{\infty, CH_3CH_2COO^-},
\end{equation}
kde $\Lambda_\ri{\infty, H^+}$ je limitní molární vodivost vodíkových kationtů a $\Lambda_\ri{\infty, CH_3CH_2COO^-}$ je limitní molární vodivost aniontů kyseliny propionové.\\
Disociační konstanta slabého elektrolytu, $K_\ri{A}$, závisí na stupni disociace, $\alpha$, podle vztahu
\begin{equation}
	K_\ri{A} = \dfrac{c_\ri{rel}\cdot \alpha^2}{1-\alpha},
\end{equation}
kde $c_\ri{rel}$ je relativní koncentrace elektrolytu v roztoku. Tuto relativní koncentraci získáme podělením molární koncentrace elektrolytu $c_\ri{CH_3CH_2COOH}$ standardní molární koncentrací $c^\ominus = 1\jd{mol\cdot dm^{-3}}$.\\
Dosazením rovnice (1) do rovnice (3) získáme Ostwaldův zřeďovací zákon
\begin{equation}
	K_\ri{A} = \dfrac{c_\ri{rel}\dfrac{\Lambda}{\Lambda_\infty}}{1-\dfrac{\Lambda}{\Lambda_\infty}},
\end{equation}
jehož úpravou dostaneme závislost reciproké vodivosi, $\dfrac{1}{\Lambda}$, na součinu vodivosti a relativní koncentrace roztoku, $c_\ri{rel}\cdot \Lambda$, tedy
\begin{equation}
	\dfrac{1}{\Lambda} = \dfrac{1}{K_\ri{A}\cdot \Lambda_\infty^2} c_\ri{rel}\cdot \Lambda + \dfrac{1}{\Lambda_\infty}.
\end{equation}
Při samotném vyhodnocování získaných dat tedy vyneseme do grafu závislost $\dfrac{1}{\Lambda}$ na $c_\ri{rel}\cdot \Lambda$, kterou proložíme přímku, z jejíž směrnice pak spočítáme hodnotu disociační konstanty.\\
Během úlohy nebyla měřena přímo vodivost roztoku, $\Lambda$, ale jeho konduktivita, neboli měrná vodivost. Vztah mezi vodivostí a konduktivitou je popsán rovnicí
\begin{equation}
	\Lambda = \dfrac{\kappa}{c_\ri{CH_3CH_2COOH}},
\end{equation}
kde $\kappa$ je konduktivita roztoku a $c$ je koncentrace kyseliny propionové v měřeném roztoku. Protože i čistá voda vede díky autoprotolýze elektrický proud, jsou hodnoty konduktivity kyseliny propionové díky přítomnosti vody nepřesné. Pro výpočet vodivosti kyseliny propionové tak konduktivitu kyseliny propionové získáme odečtením konduktivity čisté vody $\kappa_\ri{H_2O}$ od naměřené konduktivity roztoku $\kappa$. Po dosazení tohoto rozdílu do vztahu (6) získáme pro výpočet vodivosti kyseliny propionové vztah
\begin{equation}
	\Lambda = \dfrac{\kappa -\kappa_\ri{H_2O}}{c_\ri{CH_3CH_2COOH}}.
\end{equation}
\section*{Postup}
\noindent Nejprve byl zapnut termostat a nastaven na teplotu $25\jd{^\circ C}$. Do hnědé skleněné nádobky byla nalita destilovaná voda a uzavřená nádobka byla vložena do termostatu. Do vodivostní nádobky bylo napipetováno 40,0\jd{ml} 0,1013M kyseliny propionové. Konduktometr byl zapnut a po stisknutí tlačítka SAMPLE byla vodivostní cela vložena do nádobky s kyselinou propionovou. Vodivostní cela byla ponechána vytemperovat po dobu 15\jd{min} a poté byla stiskem tlačítka PRINT změřena konduktivita $\kappa$ roztoku kyseliny propionové. Poté bylo z vodivostní nádobky odebráno 20\jd{ml} roztoku a přidáno 20\jd{ml} vytemperované destilované vody. Vodivostní celou byl roztok promíchán a opět byla změřena konduktivita pro roztok kyseliny propionové o poloviční koncentraci. Stejným způsobem byl roztok zředěn a změřena jeho konduktivita ještě osmkrát, takže jako poslední byl měřen roztok o koncentraci rovné $\frac{1}{512}$ původní koncentrace. Poté byla vodivostní nádobka i cela umyta destilovanou vodou a byla změřena konduktivita  čisté vody, jejíž hodnota byla $2,30\jd{\upmu S\cdot cm^{-1}}$. Naměřené hodnoty konduktivity kyseliny propionové jsou v tabulce č. 1.
\begin{center}
	\noindent Tabulka č. 1: Koncentrace kyseliny propionové a naměřené hodnoty konduktivity při jednotlivých měřeních\\
	\begin{tabular}{c|c|c}
		Č. měření & $c_\ri{CH_3CH_2COOH}/\jd{mol\cdot dm^{-3}}$ & $\kappa/\jd{\upmu S\cdot cm^{-1}}$\\
		\hline
		1 & 0,1013 & 438\\
		2 & 0,05065 & 374,1\\
		3 & 0,02533 & 221,9\\
		4 & 0,01267 & 160\\
		5 & 0,006335 & 112,5\\
		6 & 0,003168 & 75,8\\
		7 & 0,001584 & 52,1\\
		8 & 0,0007920 & 39,5\\
		9 & 0,0003960 & 26,55\\
		10 & 0,0001980 & 17,44\\
	\end{tabular}
\end{center}
\section*{Vyhodnocení výsledků}
Získaná data byla vyhodnocena v programu Origin. Podle rovnice (7) byla spočítána hodnota vodivosti pro jednotlivé koncentrace kyseliny propionové. Pro měření č. 1 získáme po dosazení výpočet
$$\Lambda = \dfrac{0,0483\jd{S\cdot m^{-1}}-0,000230\jd{S\cdot m^{-1}}}{101,3\jd{mol\cdot m^{-3}}} = 0,000430\jd{S\cdot m^2\cdot mol^{-1}}.$$
Hodnoty vodivosti pro všechna měření jsou v tabulce č. 2.\\
Limitní molární vodivost roztoku byla získána dosazením do rovnice (2). Hodnota $\Lambda_\ri{\infty, CH_3CH_2COO^-}$ je podle [1] rovna $0,034965\jd{S\cdot m^2\cdot mol^{-1}}$. Hodnota $\Lambda_\ri{\infty, H^+}$ je podle [1] rovna $0,00358\jd{S\cdot m^2\cdot mol^{-1}}$. Po dosazení tedy získáme výpočet
$$\Lambda_\infty = 0,034965 + 0,00358 = 0,03855\jd{S\cdot m^2\cdot mol^{-1}}.$$
Následně byly pro každé měření spočítány hodnoty $\dfrac{1}{\Lambda}$ a $c_\ri{rel}\cdot \Lambda$. Pro měření č. 1 po dosazení získáme výpočty
$$\dfrac{1}{\Lambda} = \dfrac{1}{0,000430} = 2,32\cdot 10^3\jd{mol\cdot S^{-1}\cdot m^{-2}}$$
a $$c_\ri{rel}\cdot \Lambda = 0,1013\cdot 0,000430 = 0,0000436\jd{S\cdot m^2\cdot mol^{-1}}.$$
Tyto hodnoty pro všechna měření jsou uvedeny v tabulce č. 2.\\
\begin{center}
	\noindent Tabulka č. 2: Spočítané hodnoty pro jednotlivá měření\\
	\begin{tabular}{c|c|c|c}
		Č. měření & $\Lambda/\jd{S\cdot m^2 \cdot mol^{-1}}$ & $\Lambda^{-1}/\jd{mol\cdot m^{-2} \cdot S^{-1}}$ & $c_\ri{rel}\cdot \Lambda/\jd{S\cdot m^2 \cdot mol^{-1}}$\\
		\hline
		1 & 0,000430 & $2,32\cdot 10^3$ & $4,36\cdot 10^{-5}$\\
		2 & 0,0007341 & 1362 & $3,718\cdot 10^{-5}$\\
		3 & 0,0008670 & 1153 & $2,196\cdot 10^{-5}$\\
		4 & 0,00124 & 803 & $1,58\cdot 10^{-5}$\\
		5 & 0,001740 & 574,9 & $1,102\cdot 10^{-5}$\\
		6 & 0,00232 & 431 & $7,35\cdot 10^{-6}$\\
		7 & 0,00314 & 318 & $4,98\cdot 10^{-6}$\\
		8 & 0,00470 & 213 & $3,72\cdot 10^{-6}$\\
		9 & 0,006124 & 163 & $2,43\cdot 10^{-6}$\\
		10 & 0,007646 & 131 & $1,51\cdot 10^{-6}$\\
	\end{tabular}
\end{center}
Poté byla do grafu č. 1 vynesena závislost $\dfrac{1}{\Lambda}$ na $c_\ri{rel}\cdot \Lambda$. Vzhledem k velké odlehlosti bodu získaného pro měření č. 2 byla tato data pro další vyhodnocení vyloučena. Zbylých devět bodů bylo proloženo přímkou, pro jejíž směrnici, v tabulce u grafu č.1  označenou jako Slope, vzhledem k rovnici (5) platí vztah 
\begin{equation}
	5,21\cdot 10^7 \pm 8\cdot 10^5 = \dfrac{1}{K_\ri{A}\cdot \Lambda_\infty^2},
\end{equation}
ze kterého pro disociační konstantu po dosazení vychází výpočet
$$K_\ri{A} = \dfrac{1}{	5,21\cdot 10^7 \pm 8\cdot 10^5 \cdot 0,03855^2} = (1,29\pm 0,02)\cdot 10^{-5}.$$
Pro logaritmický tvar disociační konstanty platí rovnice
\begin{equation}
	\ri{p}K_\ri{A} = -\log K_\ri{A},
\end{equation}
ze které po dosazení získáme výpočet
$$\ri{p}K_\ri{A} = -\log (1,29\pm 0,02)\cdot 10^{-5} = 4,89\pm 0,01.$$
\section*{Diskuse}
Hodnota logaritmického tvaru disociační konstanty kyseliny propionové je podle [1] rovna 4,87. Námi stanovená hodnota $4,89\pm 0,01$ je tabelované hodnotě velmi blízká a měření tedy bylo docela přesné. Malá odchylka od tabelované hodnoty mohla být způsobena nepřesným pipetováním při ředění roztoku.
\section*{Závěr}
Hodnota p$K_\ri{A}$ kyseliny propionové je $4,89\pm 0,01$.
\section*{Zdroje}
[1] J. Vohlídal, A. Julák, K. Štulík: Chemické a analytické tabulky, Praha 1999
\end{document}